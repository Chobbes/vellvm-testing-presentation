\documentclass{beamer}

\usepackage{hyperref}
\usepackage{listings}
\usepackage{color}

\usepackage[backend=bibtex]{biblatex}
\addbibresource{citations.bib}

% http://tex.stackexchange.com/questions/68080/beamer-bibliography-icon
\setbeamertemplate{bibliography item}{%
  \ifboolexpr{ test {\ifentrytype{book}} or test {\ifentrytype{mvbook}}
    or test {\ifentrytype{collection}} or test {\ifentrytype{mvcollection}}
    or test {\ifentrytype{reference}} or test {\ifentrytype{mvreference}} }
    {\setbeamertemplate{bibliography item}[book]}
    {\ifentrytype{online}
       {\setbeamertemplate{bibliography item}[online]}
       {\setbeamertemplate{bibliography item}[article]}}%
  \usebeamertemplate{bibliography item}}

\defbibenvironment{bibliography}
  {\list{}
     {\settowidth{\labelwidth}{\usebeamertemplate{bibliography item}}%
      \setlength{\leftmargin}{\labelwidth}%
      \setlength{\labelsep}{\biblabelsep}%
      \addtolength{\leftmargin}{\labelsep}%
      \setlength{\itemsep}{\bibitemsep}%
      \setlength{\parsep}{\bibparsep}}}
  {\endlist}
  {\item}

% Colours for beamer.
\setbeamercolor{frametitle}{fg=orange}
\setbeamertemplate{itemize item}{\color{orange}$\blacksquare$}
\setbeamertemplate{itemize subitem}{\color{orange}$\blacktriangleright$}

% Colours for syntax highlighting
\definecolor{syntax_red}{rgb}{0.7, 0.0, 0.0} % For strings
\definecolor{syntax_green}{rgb}{0.15, 0.5, 0.25} % For comments
\definecolor{syntax_orange}{rgb}{0.7, 0.4, 0.2} % For keywords


% Haskell settings for lstlisting
\lstset{language=Haskell,
basicstyle=\ttfamily\tiny,
keywordstyle=\color{syntax_orange}\bfseries,
stringstyle=\color{syntax_red},
commentstyle=\color{syntax_green},
numbers=none,
numberstyle=\color{black},
stepnumber=1,
frame=single,
breaklines=true,
numbersep=10pt,
tabsize=4,
showspaces=false,
showstringspaces=false}

\author{
  Beck, Calvin\\
  \href{mailto:hobbes@seas.upenn.edu}{hobbes@seas.upenn.edu}
  \and\\~\\
  Zakowski, Yannick\\
  \href{mailto:zakowski@seas.upenn.edu}{zakowski@seas.upenn.edu}
}

\begin{document}

\begin{frame}
  \frametitle{QuickChecking LLVM}
  \maketitle
\end{frame}

\section{Introduction}

\begin{frame}
  \begin{itemize}
  \item Use properties based testing to test Vellvm
  \end{itemize}
\end{frame}

\begin{frame}
  \frametitle{A Quick Refresher}
  What is properties based testing? (PBT)

  \begin{itemize}
  \item Basic idea is to test properties of your program using
    automatically generated data.
  \item A property is something like ``the sort function produces
    sorted lists'' or ``if I insert an element $v$ at key $k$ in a
    map, I'll get back $v$ if I look up $k$''.
  \item Test data can be generated in different ways
    \begin{itemize}
    \item You can enumerate all test cases
    \item You can randomly pick the test cases
    \item You can mutate a suite of test cases (TODO: cite fuzzchick)
      \pause
    \item Some mix of the above
    \end{itemize}
  \end{itemize}
  
  \pause
  
  Our focus is mostly on random testing---it's shockingly effective,
  though we do anticipate mixing in enumeration in order to get
  backtracking will be effective for handling complex constraints.
\end{frame}

\begin{frame}
  What is Vellvm?

  \begin{itemize}
  \item Semantics for LLVM in Coq
  \item An interpreter for LLVM
  \item One of the primary goals is to produce verified optimization
    passes for LLVM
  \end{itemize}
\end{frame}

\begin{frame}
  So, why test compilers?

  \begin{itemize}
  \item compilers are complex, with many components
          \begin{itemize}
          \item parsers, lexers, frontends, backends, middlends, oh my.
          \item lots of things that have to work correctly together.
          \end{itemize}
  \item foundational --- nearly all software depends on one in some
    way or another.
    \begin{itemize}
    \item Getting the compiler wrong means any program that was
      influenced by this compiler could be wrong too.
    \end{itemize}
  \end{itemize}
\end{frame}

\begin{frame}
  What to test with compilers

  % TODO: this slide may not survive. was initially mentioned the
  % components of compilers (front end, middle end, back end), and that
  % each of these might involve different testing methods + that we're
  % not really focusing on testing parsers or anything.
  %
  % Though, ultimately we're testing with respect to llc's full chain
  % of compilation.
  
\end{frame}

\begin{frame}
  \frametitle{}

  How do we know a compiler is working correctly?

  \begin{itemize}
  \item Unit test various functions
  \item Manually write down test programs, building up a huge library
    of tests.
  \item Write down an end-to-end proof of correctness
    TODO: make sure to link this to QuickCheck
  \end{itemize}

  How you test may depend on the component(s) of the compiler being
  tested.

  % Different things may be easier to do with different parts of the
  % compiler. For instance if the compiler has an internal
  % representation for an intermediate language, this might be harder
  % to test just by building up a sample collection of programs.

  \pause

  \begin{itemize}
  \item Manual unit tests, and suites of test programs can be really
    helpful for testing specific behaviours, but can be really tedious
    to create.
  \item Proofs are obviously very hard to write
    \begin{itemize}
    \item Plus you can only write a proof if your compiler is correct
      to begin with...
    \end{itemize}
  \end{itemize}

  \pause

  Might be nice to automatically generate test cases and run them?

\end{frame}

\begin{frame}

  Generating programs seems hard.

  \begin{itemize}
  \item Lots of constraints
    \begin{itemize}
    \item Need appropriate syntax (if generating syntax)
    \item Needs to typecheck
    \item Needs to be a program
    \end{itemize}
  \item There are many complex, undecidable constraints...
    \begin{itemize}
    \item You might want programs to terminate
    \item Memory safety
    \item Avoid undefined behaviours
    \end{itemize}
  \end{itemize}
\end{frame}

\begin{frame}
  Furthermore knowing what a program is seems hard...

  \begin{itemize}
  \item Many undefined behaviours in languages like C
  \item The semantics of the language may not be formally specified
    (or even exist)
    \begin{itemize}
    \item Makes it hard to test
    \end{itemize}
  \end{itemize}
\end{frame}

\begin{frame}

  Despite all of these issues tools like C smith seem to do a really
  good job.

  \begin{itemize}
  \item C smith is a tool for testing C compilers by randomly
    generating C programs.
  \item These C programs are run through different C compilers, and
    the results are compared.
  \end{itemize}

  C smith has found hundreds of defects in real world C compilers.
\end{frame}

\begin{frame}
  Ultimately we want to test Vellvm.
\end{frame}

\begin{frame}
  Why QuickChick in Coq? Why not just prove things?

  \begin{itemize}
  \item Specifications are hard. How do you know you're spec is
    correct?
    \begin{itemize}
    \item Theorems are hard to state
    \end{itemize}
  \item Proofs are hard.
  \end{itemize}

  QuickChick, in a verification setting, is arguably not used to
  ensure the correctness of the program. It's there to ease
  development.
\end{frame}

\begin{frame}
  Why QuickChick in Vellvm?

  \begin{itemize}
  \item What do we want to test?
  \item Is this just for us, or is it also for users?
  \item Current idea is to do differential testing with clang just to
    see if we differ.
  \item Ultimately we want to test this with 
  \end{itemize}
\end{frame}

\begin{frame}
  So QuickChecking Vellvm seems like a good idea, which means we have
  to generate programs. Why might that be hard?
\end{frame}

\begin{frame}
  Current approach. Very limited.

  \begin{itemize}
  \item What are we testing?
  \item
  \end{itemize}
\end{frame}

\begin{frame}
  \frametitle{Questions?}

\end{frame}

\begin{frame}
  \frametitle{References}

  \nocite{*}
  \printbibliography
\end{frame}

\end{document}
