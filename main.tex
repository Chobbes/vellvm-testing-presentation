\documentclass{beamer}
\beamertemplatenavigationsymbolsempty

\usepackage{hyperref}
\usepackage{listings}
\usepackage{color}

\usepackage{preamble}

\usepackage[backend=bibtex]{biblatex}
\addbibresource{citations.bib}

% http://tex.stackexchange.com/questions/68080/beamer-bibliography-icon
\setbeamertemplate{bibliography item}{%
  \ifboolexpr{ test {\ifentrytype{book}} or test {\ifentrytype{mvbook}}
    or test {\ifentrytype{collection}} or test {\ifentrytype{mvcollection}}
    or test {\ifentrytype{reference}} or test {\ifentrytype{mvreference}} }
    {\setbeamertemplate{bibliography item}[book]}
    {\ifentrytype{online}
       {\setbeamertemplate{bibliography item}[online]}
       {\setbeamertemplate{bibliography item}[article]}}%
  \usebeamertemplate{bibliography item}}

\defbibenvironment{bibliography}
  {\list{}
     {\settowidth{\labelwidth}{\usebeamertemplate{bibliography item}}%
      \setlength{\leftmargin}{\labelwidth}%
      \setlength{\labelsep}{\biblabelsep}%
      \addtolength{\leftmargin}{\labelsep}%
      \setlength{\itemsep}{\bibitemsep}%
      \setlength{\parsep}{\bibparsep}}}
  {\endlist}
  {\item}

% Colours for beamer.
\setbeamercolor{frametitle}{fg=orange}
\setbeamertemplate{itemize item}{\color{orange}$\blacksquare$}
\setbeamertemplate{itemize subitem}{\color{orange}$\blacktriangleright$}

% Colours for syntax highlighting
\definecolor{syntax_red}{rgb}{0.7, 0.0, 0.0} % For strings
\definecolor{syntax_green}{rgb}{0.15, 0.5, 0.25} % For comments
\definecolor{syntax_orange}{rgb}{0.7, 0.4, 0.2} % For keywords


% Haskell settings for lstlisting
\lstset{language=Haskell,
basicstyle=\ttfamily\tiny,
keywordstyle=\color{syntax_orange}\bfseries,
stringstyle=\color{syntax_red},
commentstyle=\color{syntax_green},
numbers=none,
numberstyle=\color{black},
stepnumber=1,
frame=single,
breaklines=true,
numbersep=10pt,
tabsize=4,
showspaces=false,
showstringspaces=false}

\author{
  Beck, Calvin\\
  \href{mailto:hobbes@seas.upenn.edu}{hobbes@seas.upenn.edu}
  \and\\~\\
  Zakowski, Yannick\\
  \href{mailto:zakowski@seas.upenn.edu}{zakowski@seas.upenn.edu}
}
\setbeamertemplate{footline}[frame number]

\begin{document}

\begin{frame}
  \frametitle{QuickChecking LLVM}
  \maketitle
\end{frame}

\section{Introduction}

\begin{frame}

  Our goal: using \emph{property-based testing} to ``test'' Vellvm.

  This talk will be structured roughly as follows:
  
  \begin{itemize}
  \item Brief overview of the Vellvm project
  \item Quick Review of QuickCheck / QuickChick
  \item Quick discussion about testing compilers
  \item Csmith
  \item Testing Vellvm using QuickChick generators inspired by Csmith
  \end{itemize}

  Feel free to ask questions. I'm not sure I can have the chat up at
  the same time, so please feel free to {\bf shout}.
\end{frame}

\section{Overview of Vellvm}

\begin{frame}
  What is llvm?

   \begin{itemize}
    \item LLVM is an intermediate representation (IR) used by many
      compilers.
    \item ``universal assembly language''
    \item Kind of a mix of an assembly language and C.
      % TODO: Insert an example of LLVM code here?
   \end{itemize}

   What is Vellvm?
   \begin{itemize}
   \item A formal semantics for IR in Coq;
   \item an interpreter for IR derived from the semantics;
   \item a family of projects in development aiming to leverage this semantics.
   \end{itemize}

\end{frame}

\begin{frame}
  \frametitle{What does Vellvm hope to do?}

  \begin{itemize}
  \item Some goals:
    \begin{itemize}
    \item Accurately reflect the ``real'' LLVM
    \item Be useful for constructing verified optimization passes
    \item Maybe have a back-end in the long term to be a part of
      verified compilation chains.
    \end{itemize}

  \item We want to answer this question:
    
    \begin{itemize}
    \item Could properties based testing help Vellvm achieve these
      goals?
    \end{itemize}
  \end{itemize}
\end{frame}

\section{QuickCheck Review}

\begin{frame}
  \frametitle{A Quick(Check) Refresher}

  What is properties based testing? (PBT)

  \begin{itemize}
  \item ``What QuickCheck does''
  \item Basic idea is to test properties of your program using
    automatically generated data.
    \begin{itemize}
    \item Nicely separates the property you care about from specific
      test data.
    \end{itemize}
  \item A property is something like:
    \begin{itemize}
    \item {\tt is\_sorted (sort l)}
    \item {\tt lookup k (insert k v m) == v}
    \end{itemize}
    % ``the sort function produces
    % sorted lists'' or ``if I insert an element $v$ at key $k$ in a
    % map, I'll get back $v$ if I look up $k$''.
  \item Must generate ``test data'' to fill in variables ({\tt l}, {\tt k}, {\tt v}, {\tt m}):
    \begin{itemize}
    \item Enumerate all possible values
    \item Randomly generate values
    \item You can mutate a suite of test data~\cite{10.1145/3360607}
    \item Some mix of the above
    \end{itemize}
  \end{itemize}
  
  \pause
  
  Our focus is mostly on random testing---it's shockingly effective,
  though we may want to use backtracking in the future.
  % TODO: does this belong here or does it just seem irrelevant.
\end{frame}

\begin{frame}
  \frametitle{Example of QuickChick}

  % Example of a generator + maybe running QuickChick

\end{frame}

\begin{frame}
  \frametitle{Why PBT in a Theorem Prover?}

  Why not just prove things?

  \begin{itemize}
  \item Proving is hard.
    \pause
  \item Proving false things is even harder.
  \end{itemize}

  QuickChick can really help you work out the kinks in your
  implementation \emph{AND} specification before trying to prove something.
\end{frame}

\section{Discussion on testing compilers}

\begin{frame}
  \frametitle{Why test compilers?}

  % TODO: Not sure that this slide adds much. maybe cut it.
  All software should be tested, but there's good reason for compilers
  to be well tested.

  \begin{itemize}
  \item compilers are complex, with many components
          \begin{itemize}
          \item parsers, lexers, front-ends, back-ends, middle-ends, oh my.
          \item lots of things that have to work correctly together.
          \end{itemize}
  \item foundational --- nearly all software relies on one somewhere.
    \begin{itemize}
    \item Getting the compiler wrong means any program that was
      influenced by this compiler could be wrong too.
    \end{itemize}
  \item Compiler bugs are hard to spot, and hard to debug.
  \end{itemize}
\end{frame}

% \begin{frame}
%   What to test with compilers

%   % TODO: this slide may not survive. was initially mentioned the
%   % components of compilers (front end, middle end, back end), and that
%   % each of these might involve different testing methods + that we're
%   % not really focusing on testing parsers or anything.
%   %
%   % Though, ultimately we're testing with respect to llc's full chain
%   % of compilation.
  
% \end{frame}

\begin{frame}
  \frametitle{Testing compilers}
  % TODO: I don't really like this slide.
  % Should be motivating PBT of compilers. Want to get to the point of
  % "it would be really nice to randomly generate programs, can we do
  % that? Csmith has found success doing this, maybe we can apply
  % what it does to LLVM, here's how Csmith works, here's our rough
  % structure for generating LLVM so far (not as fancy yet).
  %
  % I think I have changed the structure enough that I am okay with
  % this slide now.

  Lots of ways to test a compiler:

  \begin{itemize}
  \item Unit test various functions
  \item Manually write down test programs, building up a huge library
    of tests.
  \item Write down an end-to-end proof of correctness
    \begin{itemize}
    \item CompCert
    \end{itemize}
    % TODO: make sure to link this to QuickCheck
  \end{itemize}

  % Probably a tangent that I don't need.
  % How you test may depend on the component(s) of the compiler being
  % tested.

  % TODO:
  % Different things may be easier to do with different parts of the
  % compiler. For instance if the compiler has an internal
  % representation for an intermediate language, this might be harder
  % to test just by building up a sample collection of programs.
\end{frame}

\begin{frame}
  \frametitle{Is there a better way?}

  All of these are useful... But...
  
  \begin{itemize}
  \item Manual unit tests, and suites of test programs can be really
    helpful for testing specific behaviours, but can be really tedious
    to create.
    \begin{itemize}
    \item End up testing with a limited number of cases
    \end{itemize}
  \item Proofs are obviously very hard to write
    \begin{itemize}
    \item Plus you can only write a proof if your compiler is correct
      to begin with... % TODO: make sure this is connected with QuickChick.
    \end{itemize}
  \end{itemize}

  Can we augment these test suites with PBT? Can we automatically
  generate a large suite of test programs for our compiler?

\end{frame}

\begin{frame}

  Generating programs seems hard.

  \begin{itemize}
  \item Lots of constraints
    \begin{itemize}
    \item Need appropriate syntax (if generating syntax)
    \item Needs to typecheck
    \item Needs to be a program
    \end{itemize}
  \item There are many complex, undecidable constraints...
    \begin{itemize}
    \item You might want programs to terminate
    \item Memory safety
    \item Avoid undefined behaviours
    \end{itemize}
  \end{itemize}
\end{frame}

\begin{frame}
  Furthermore knowing what a program is seems hard...

  \begin{itemize}
  \item Many undefined behaviours in languages like C
  \item Language is ill specified
    \begin{itemize}
    \item Semantics most likely informal, if specified at all
    \item Hard to know *what* to test
    \end{itemize}
  \end{itemize}
\end{frame}

\section{Csmith}

\begin{frame}

  Despite all of these issues, tools like Csmith seem to do a really
  good job.

  \begin{itemize}
  \item Csmith is a tool for testing C compilers by randomly
    generating C programs.
  \item These C programs are run through different C compilers, and
    the results are compared.
  \end{itemize}

  Csmith has found hundreds of defects in real world C compilers.
\end{frame}

\begin{frame}
    \frametitle{What does Csmith do?}
  What does Csmith do, and how can we adapt that to test Vellvm?

  \begin{itemize}
  \item Production rules based on the grammar of the programming
    language
  \item Each production is chosen based on a specific probability
  \item Csmith filters out choices that are invalid in the current
    context...
    \begin{itemize}
    \item Continue / break can only appear in a loop, etc.
    \end{itemize}
  \item Csmith generates variables and functions for calls on the fly
    (or selects from existing ones)
  \end{itemize}
\end{frame}

\begin{frame}
  \frametitle{Safety and Csmith}
  \begin{itemize}
  \item Csmith also uses a collection of safety checks to make sure
    the generated code is valid, and if not backtracks to generate
    something else.
    \begin{itemize}
    \item Possibly a good use for backtracking in QuickChick
    \end{itemize}
  \item Csmith has other safety features...
    \begin{itemize}
    \item Uses a library of safe arithmetic operators with checks for
      overflows, etc.
    \item Indexes to arrays are only modified in the increment part of
      {\tt for} loops, forced to be in bounds with modulo, etc.
    \end{itemize}
  \end{itemize}

\end{frame}

% I think this is irrelevant
% \begin{frame}
%   \frametitle{Termination and Csmith}

%   Csmith just uses timeouts in its test harness.

%   \begin{itemize}
%   \item I'm a little unclear whether it just discards timeouts, or
%     actually uses the results.
%   \item Always safe to discard, anyway.
%   \end{itemize}

%   Some of these concerns are likely mitigated by generating idiomatic
%   loops, such as loops iterating over array elements, making
%   termination much more likely.
% \end{frame}
% \begin{frame}
%   Ultimately we want to test Vellvm.
% \end{frame}

\section{QuickChecking Vellvm}

\begin{frame}
  Why QuickChick in Vellvm?

  \begin{itemize}
  \item What do we want to test?
  \item Is this just for us, or is it also for users?
  \item Current idea is to do differential testing with clang, and see
    where we disagree.
    \begin{itemize}
    \item Find issues with the spec!
    \end{itemize}
  \item Ultimately we want to be able provide the generators so users
    can check ``does my optimization pass preserve the semantics of programs''.
  \end{itemize}
\end{frame}

\begin{frame}
  So QuickChecking Vellvm seems like a good idea, which means we have
  to generate programs.

  \begin{itemize}
  \item Non-trivial, but we can follow Csmiths footsteps.
  \item LLVM actually has a huge grammar with lots of little
    constraints everywhere
    \begin{itemize}
    \item We will only cover a subset of a subset for now
    \end{itemize}
  \end{itemize}
\end{frame}

\begin{frame}
  The generators we have built are currently limited.

  \begin{itemize}
  \item Straight line programs
    \begin{itemize}
    \item No backwards edges in the CFG
    \item Major simplification, but for now we don't have to worry
      about termination.
    \end{itemize}
  \item Only integers for now
  \item Only check the observable parts of the program
    \begin{itemize}
    \item Check the return value, and nothing else
    \item (We have no I/O anyway)
    \end{itemize}
  \end{itemize}
\end{frame}

\begin{frame}
  \frametitle{Overview of the generators}

  \begin{itemize}
  \item Various different type generators for slightly different
    constraints everywhere.
  \item Expression generator is straightforward.
  \item Instructions...
  \item CFG generator
  \end{itemize}

\end{frame}

\begin{frame}
  \frametitle{Future Work}

  \begin{itemize}
  \item Underspecified values and nondeterminism are crucial for
    optimizations in LLVM. Unclear how to compare two programs under
    this, though. Maybe incorporate Z3?
  \item Better more complex generator.
  \item Shrinking
    \begin{itemize}
    \item Csmith uses delta-debugging, and has to be careful about
      shrinking causing undefined behaviour...
    \item We will also have to worry about this, though in principle
      Vellvm should be able to report if UB occurs during execution.
    \end{itemize}
  \end{itemize}
\end{frame}

\begin{frame}
  \frametitle{Questions?}

\end{frame}

\begin{frame}
  \frametitle{References}

  \nocite{*}
  \printbibliography
\end{frame}

\end{document}
